\documentclass[lang=cn,newtx,10pt,scheme=chinese]{elegantbook}

\title{数学分析}
\subtitle{笔记分享}

\author{An Ordinary Math Enthusiast}
\institute{CHINA}
\date{2024/7/21}
\version{1.0.0}
\bioinfo{自定义}{这是一份笔记分享,有错误还望指出批评}

 \extrainfo{梦想可以大,但第一步总是小\\Dreams can be big, but the first step is always small.}

\setcounter{tocdepth}{3}

\logo{su5.png}
\cover{su4.png}

% 本文档命令
\usepackage{array}
\newcommand{\ccr}[1]{\makecell{{\color{#1}\rule{1cm}{1cm}}}}

% 修改标题页的橙色带
\definecolor{customcolor}{RGB}{32,178,170}
\colorlet{coverlinecolor}{customcolor}
\usepackage{cprotect}

\addbibresource[location=local]{reference.bib} % 参考文献,不要删除

\begin{document}

\maketitle
\frontmatter

\tableofcontents

\mainmatter

\chapter{六大定理的相互证明}

\begin{introduction}
  \item 确界定理(实数系连续性定理)~\ref{def:int1}
  \item 单调有界定理~\ref{thm:fubi2}
  \item Cauchy定理~\ref{pro:max3}
  \item 区间套定理~\ref{property:cauchy3}
  \item 聚点定理 ~\ref{tabe222}
  \item 有限覆盖定理(Heine-Borel定理)~\ref{pro:js5}
\end{introduction}

\begin{change}
  \item 确界定理(实数系连续性定理):非空有上界的实数集必有上确界,非空有下界的实数集必有下确界。
  \item 单调有界定理:在实数系中,单调且有界的数列必定收敛。
  \item Cauchy定理:数列 \(\{x_n\}\) 收敛的充要条件是 \(\{x_n\}\) 是基本(柯西)数列。
  \item 区间套定理:若 \(\{[a_n, b_n]\}\) 形成一个闭区间套,则存在唯一的实数 \(\xi\) 属于所有的闭区间 \([a_n, b_n]\),\(n = 1, 2, 3, \cdots\),且 \( \xi \) 属于所有这些闭区间,并且有 \( \lim_{n \to \infty} b_n \) = \( \lim_{n \to \infty} a_n  \) =\( \xi \) 
  \item 聚点定理、魏尔斯特拉斯定理、致密性定理、Bolzano-Weierstrass定理:
    \begin{itemize}
      \item 聚点定理:在一个有界序列中,至少存在一个聚点。
      \item 魏尔斯特拉斯定理:每个有界数列都有至少一个聚点。
      \item 致密性定理:一个集合是致密的当且仅当它的每一个开覆盖都有有限子覆盖。
      \item Bolzano-Weierstrass定理:任何有界数列必有一个收敛的子序列。
    \end{itemize}
\begin{enumerate}
  \item \textbf{聚点定理}:
    \begin{itemize}
      \item \textbf{定义}:在一个有界序列中,至少存在一个聚点。
      \item \textbf{说明}:聚点是指序列中某一子序列的极限。如果一个序列是有界的,那么它至少有一个聚点。
      \item \textbf{应用}:该定理说明了有界序列在某种意义上不会“散开”,一定存在某些点是序列的极限点。
    \end{itemize}
    
  \item \textbf{魏尔斯特拉斯定理}(也称为\textbf{有界性原理}):
    \begin{itemize}
      \item \textbf{定义}:每个有界数列都有至少一个聚点。
      \item \textbf{说明}:这个定理与聚点定理基本一致,它强调了有界数列一定存在至少一个聚点。
      \item \textbf{应用}:该定理是序列收敛性分析中的基础,特别是对于有界数列的性质研究。
    \end{itemize}
    
  \item \textbf{致密性定理}:
    \begin{itemize}
      \item \textbf{定义}:一个集合是致密的当且仅当它的每一个开覆盖都有有限子覆盖。
      \item \textbf{说明}:这个定理描述了致密集(通常称为紧致集或紧集)的特性。一个集合如果每一个开覆盖(即用开集覆盖整个集合)都可以找到一个有限的子覆盖,那么这个集合是致密的。
      \item \textbf{应用}:致密性定理在拓扑学中非常重要,用于研究集合的极限性质和连续函数的性质。
    \end{itemize}
    
  \item \textbf{Bolzano-Weierstrass定理}:
    \begin{itemize}
      \item \textbf{定义}:任何有界数列必有一个收敛的子序列。
      \item \textbf{说明}:这个定理指出,有界数列总是可以找到一个收敛的子序列。这意味着有界数列在某种程度上总是可以提取出一个收敛的部分。
      \item \textbf{应用}:Bolzano-Weierstrass定理在分析学中广泛应用,特别是在证明各种关于收敛性的命题时。
    \end{itemize}
\end{enumerate}

总结:
\begin{itemize}
  \item \textbf{聚点定理}和\textbf{魏尔斯特拉斯定理}主要关注有界数列的聚点存在性。
  \item \textbf{致密性定理}关注的是集合的覆盖性质,是一个拓扑学定理。
  \item \textbf{Bolzano-Weierstrass定理}进一步说明了有界数列不仅有聚点,而且必有一个收敛的子序列。
\end{itemize}

它们各自从不同的角度描述了数列和集合的性质,在数学分析和拓扑学中都有着广泛的应用。

  \item 有限覆盖定理(Heine-Borel 定理):
    \begin{itemize}
      \item \textbf{定义}:一个集合是紧的当且仅当它是闭且有界的;设 $H$ 为闭区间 $[a,b]$ 的一个(无限)开覆盖,则从 $H$ 中可选出有限个开区间来覆盖 $[a,b]$。
      \item \textbf{说明}:如果一个开覆盖 \( S \) 覆盖了闭区间 \([a, b]\),即 \([a, b] \subseteq \bigcup_{i \in I} U_i\),其中 \( U_i \) 是 \( S \) 中的开区间,则存在有限个开区间 \( U_{i_1}, U_{i_2}, \ldots, U_{i_n} \) 使得 \([a, b] \subseteq \bigcup_{j=1}^n U_{i_j}\)。
    \end{itemize}

\end{change}


\begin{note}
\begin{enumerate}
    \item \textbf{有界集}:\\
    集合 \( A \subseteq \mathbb{R} \) 称为有界的,如果存在实数 \( M \) 使得对于所有 \( x \in A \),有 \( |x| \leq M \)。即集合 \( A \) 的所有元素都被限制在一个有限的范围内。

    \item \textbf{上(下)确界}:\\
    给定一个非空的有上界集合 \( S \subseteq \mathbb{R} \),上确界(或称为最小上界) \( \sup S \) 是满足下列条件的唯一实数:
    \begin{enumerate}
        \item 对于所有 \( x \in S \),有 \( x \leq \sup S \)。
        \item 如果 \( y \) 是任意上界,则 \( \sup S \leq y \)。
    \end{enumerate}
    类似地,集合 \( S \) 的下确界(或称为最大下界) \( \inf S \) 定义为:
    \begin{enumerate}
        \item 对于所有 \( x \in S \),有 \( x \geq \inf S \)。
        \item 如果 \( z \) 是任意下界,则 \( \inf S \geq z \)。
    \end{enumerate}

    \item \textbf{数列收敛}:\\
    数列 \( \{a_n\} \) 称为收敛的,如果存在实数 \( L \) 使得对于任意 \( \epsilon > 0 \),存在正整数 \( N \) 使得当 \( n \geq N \) 时,\( |a_n - L| < \epsilon \)。此时,\( L \) 被称为数列 \( \{a_n\} \) 的极限,记作 \( \lim_{n \to \infty} a_n = L \)。

    \item \textbf{闭区间套}:\\
    一列闭区间 \( \{[a_n, b_n]\} \) 称为闭区间套,如果对于所有 \( n \) 都有 \( [a_{n+1}, b_{n+1}] \subseteq [a_n, b_n] \),且 \( \lim_{n \to \infty} (b_n - a_n) = 0 \)。根据闭区间套定理,存在唯一的实数 \( \xi \),使得 \( \xi \) 属于所有这些闭区间,即 \( \xi \in \bigcap_{n=1}^{\infty} [a_n, b_n] \)。

    \item \textbf{聚点  邻域  开集}:
    \begin{enumerate}
        \item \textbf{聚点}:\\
        如果对于任意 \( \epsilon > 0 \),在区间 \( (x - \epsilon, x + \epsilon) \) 内包含集合 \( A \) 的无限多个点,则称 \( x \) 是集合 \( A \) 的一个聚点。
        \item \textbf{邻域}:\\
        一个点 \( x \) 的 \( \epsilon \)-邻域定义为开区间 \( (x - \epsilon, x + \epsilon) \)。
        \item \textbf{开集}:\\
        集合 \( G \subseteq \mathbb{R} \) 称为开集,如果对于集合 \( G \) 中的任意一点 \( x \),存在 \( x \) 的一个邻域完全包含于 \( G \) 中。
    \end{enumerate}

    \item \textbf{基本数列}:\\
    数列 \( \{a_n\} \) 称为基本数列(或柯西数列),如果对于任意 \( \epsilon > 0 \),存在正整数 \( N \),使得当 \( m, n \geq N \) 时,有 \( |a_n - a_m| < \epsilon \)。根据柯西收敛定理,数列 \( \{a_n\} \) 收敛的充要条件是它是一个基本数列。
\end{enumerate}
\end{note}
%\newpage

\begin{theorem}[确界定理(实数系连续性定理)] \label{def:int1}
(1)非空有上界的实数集必有上确界,非空有下界的实数集必有下确界。

(2)设 $S$ 是实数集 $\mathbb{R}$ 的一个非空子集,且 $S$ 有上界,则 $S$ 在 $\mathbb{R}$ 中有一个上确界。

\[
\forall S \subseteq \mathbb{R}, S \neq \emptyset \text{ 且 } \exists u \in \mathbb{R} \text{ 使得 } \forall x \in S, x \leq u, \exists \sup S \in \mathbb{R} \text{ 使得 } \sup S = \inf \{u \in \mathbb{R} \mid \forall x \in S, x \leq u\}
\]

\end{theorem}

\section*{确界存在定理 $\Rightarrow$ 单调有界定理   }

\textbf{证明:} 设数列 $\{x_n\}$ 是单调增加且有上界,令 $S = \{x_1, x_2, \ldots\}$,根据\textcolor{red}{确界存在定理},存在 $\beta \in \mathbb{R}$,使得 $\beta = \sup S$,满足如下条件:
\begin{enumerate}
    \item $\forall x_k \in S, k \in \mathbb{N}^{+}$,有 $\beta \geq x_k$;
    \item $\forall \varepsilon > 0$,$\exists n_0 \in \mathbb{N}^{+}$ 使得 $x_{n_0} + \varepsilon > \beta$。
\end{enumerate}

于是 $\forall \varepsilon > 0$,$\exists N = n_0$,$\forall n > N$,有 $x_n + \varepsilon \geq x_N + \varepsilon > \beta > x_n > x_n - \varepsilon$。

因此 $\forall \varepsilon > 0$,$\exists N = n_0$,$\forall n > N$,有$\beta - \varepsilon<  x_n < \beta + \varepsilon$
\[
\lim_{n \to \infty} x_n = \beta。
\]

\section*{确界存在定理 $\Rightarrow$ Cauchy 收敛原理  }   

\textbf{证明:} (必要性)先证数列 $\{x_n\}$ 收敛 $\Rightarrow$ 数列 $\{x_n\}$ 是基本数列。\\
设 $\lim_{n \to \infty} x_n = a$,则 $\forall \varepsilon > 0$,$\exists N \in \mathbb{N}^{+}$,使得
\[
\forall n, m > N, \, 都有 \, |x_{n,m} - a| < \frac{\varepsilon}{2}, \, 从而 \, |x_n - x_m| \leq |x_n - a| + |x_m - a| < \varepsilon,
\]
因此数列 $\{x_n\}$ 是基本数列。

再证(充分性)数列 $\{x_n\}$ 是基本数列 $\Rightarrow$ 数列 $\{x_n\}$ 收敛。\\
设数列 $\{x_n\}$ 是基本数列,则数列 $\{x_n\}$ 有界,$\forall \varepsilon > 0$,$\exists N \in \mathbb{N}^{+}$,$\forall n, m > N$,都有 $|x_n - x_m| < \varepsilon$。取定 $\varepsilon = 1$ 时,存在 $N_1 \in \mathbb{N}^{+}$ 使得
\[
\forall n > N_1, \, |x_n - x_{N_1}| < \varepsilon,
\]
于是 $M = \max\{x_1, x_2, \ldots, x_{N_1}, |x_{N_1}| + 1\}$ 是数列 $\{x_n\}$ 的一个上界,
\[
m = \min\{x_1, x_2, \ldots, x_{N_1}, -1\}
\]
是数列 $\{x_n\}$ 的一个下界。因此数列 $\{x_n\}$ 有界。下说明数列 $\{x_n\}$ 收敛。\\
令 $S = \{x_n\}$ ,数列 $\{x_n\}$ 中小于 $x$ 的数只有有限个,显然数集 $S$ 有界,根据 \textcolor{red}{确界存在定理},数集 $S$ 必有上确界,设 $\beta = \sup S$,则 $\forall \varepsilon > 0$,$\beta - \varepsilon \in S$,$\beta + \varepsilon \notin S$,从而 $(\beta - \varepsilon, \beta + \varepsilon)$ 中含有数列 $\{x_n\}$ 中无限项,于是存在数列 $\{x_{n_k}\}$ 的子列,使得
\[
\lim_{k \to \infty} x_{n_k} = \beta,
\]
又数列 $\{x_n\}$ 是基本数列,$\forall \varepsilon > 0$,$\exists N \in \mathbb{N}^{+}$,$\forall n, m > N$,$|x_n - x_m| < \varepsilon$,取 $\varepsilon = \frac{\varepsilon}{2}$,$\forall n > N$,$k > K$ 有
\[
|x_n - \beta| \leq |x_n - x_{n_k}| + |x_{n_k} - \beta| < \varepsilon,
\]
因此数列 $\{x_n\}$ 收敛,证毕。


\section*{确界存在定理 $\Rightarrow$ 闭区间套定理}
\textbf{证明:} 设一列闭区间 $\{[a_n, b_n]\}$ 形成一个闭区间套,则数集 $S = \{a_1, a_2, \ldots\}$ ($S$ 如果有相同的数则只算作一个)有上界,而且 $b_n, n = 1, 2, 3, \ldots$ 都是数集 $S$ 的上界,存在性,根据 \textcolor{red}{确界存在定理},数集 $S$ 有上确界(最小上界),记为 $\alpha = \sup S$,于是
\begin{enumerate}
    \item $\forall n \in \mathbb{N}^{+}$,$\alpha \geq a_n$;
    \item $\forall \varepsilon > 0$,$\exists N \in \mathbb{N}^{+}$,使得 $a_N + \varepsilon > \alpha$,由 (1) 知 $a_n \leq \alpha \leq b_n$,$n = 1, 2, 3, \ldots$,由 (2) 知
\end{enumerate}

\[
\lim_{n \to \infty} a_n = \alpha, \quad \text{又} \quad \lim_{n \to \infty} (b_n - a_n) = 0, \quad \text{可知} \quad \lim_{n \to \infty} b_n = \alpha。
\]

唯一性,若另有 $\alpha'$,使得 $a_n \leq \alpha' \leq b_n,n = 1, 2, 3, \ldots$,且
\[
\lim_{n \to \infty} a_n = \lim_{n \to \infty} b_n = \alpha',
\]
由 $\lim_{n \to \infty} (b_n - a_n) = 0$,可知 $\alpha = \alpha'$,证毕。

\section*{确界存在定理 $\Rightarrow$ 聚点定理}

\textbf{证明:} 设 $S$ 是有界无穷集,根据 \textcolor{red}{确界存在定理},数集 $S$ 有上下确界,令 $\alpha = \inf S$,$\beta = \sup S$。若 $\alpha, \beta$ 其中有一点不是数集 $S$ 的孤立点,则此点必为聚点。反设,假设 $\alpha, \beta$ 其中都是数集 $S$ 的孤立点,令
\[
E = \{x \mid \text{数集 } S \text{ 中仅有有限个数小于 } x, x \in \mathbb{R}\}
\]
显然 $E$ 有上界,根据 \textcolor{red}{确界存在定理},数集 $E$ 有上确界,令 $\beta' = \sup E$,因此 $\forall \varepsilon > 0$,$\beta' + \varepsilon \notin E$,$\beta' - \varepsilon \in E$,于是
\[
(\beta' - \varepsilon, \beta' + \varepsilon) \text{ 中含有 } S \text{ 中无穷点}, \text{从而 } \beta' \text{ 是 } S \text{ 的聚点,证毕}。
\]

\section*{确界存在定理 $\Rightarrow$ 有限覆盖定理}

\textbf{证明:} 设闭区间 $[a, b]$ 在实数 $R$ 上,任取闭区间 $[a, b]$ 的一个开覆盖 $\{O_{\lambda}\}$,令
\[
D = \{x | x \in [a, b], 且 \exists [a, x] \text{ 能被 } \{O_{\lambda}\} \text{ 的有限子集覆盖}\},
\]
下说明集合 $D$ 是非空有界的。\\
$\{O_{\lambda}\}$ 是闭区间 $[a, b]$ 的一个开覆盖,则存在 $\{O_{\lambda}\}$ 的一个开区间 $O_a$,使得 $a \in O_a$,于是 $a \in D$,从而集合 $D$ 非空,又显然 $D \subseteq [a, b]$,从而集合 $D$ 是有界的,因此集合 $D$ 是非空有界的。

根据 \textcolor{red}{确界存在定理},集合 $D$ 有上确界,设 $\xi = \sup D$,下说明 $\xi = b$。假设 $\xi < b$,则 $\xi < b$,$[a, \xi]$ 能被 $\{O_{\lambda}\}$ 的有限子集 $O_{\lambda_1}, O_{\lambda_2}, \ldots, O_{\lambda_n}$ 覆盖,又 $\xi \in [a, b]$,则存在 $\{O_{\lambda}\}$ 的一个开区间 $(\alpha, \beta)$ 使得 $\xi \in (\alpha, \beta)$,任取 $x_1 \in (\xi, \beta)$,易知 $[a, x_1]$ 可被 $\{O_{\lambda_1}, O_{\lambda_2}, \ldots, O_{\lambda_n}\}$ 覆盖。从而 $x_1 \in D$,而 $x_1 > \xi$ 与 $\xi = \sup D$ 矛盾,证毕。



\begin{theorem}[单调有界定理] \label{thm:fubi2}
(1)在实数系中,单调且有界的数列必定收敛。

(2)\begin{enumerate}
    \item 如果数列 $\{a_n\}$ 是单调增加的,并且有上界,那么 $\{a_n\}$ 收敛。
    \item 如果数列 $\{a_n\}$ 是单调减少的,并且有下界,那么 $\{a_n\}$ 收敛。
\end{enumerate}

\end{theorem}

\section*{单调有界定理 $\Rightarrow$ 确界存在定理}

\textbf{证明:} 设 $S$ 是非空有上界数集,任取它的一个上界 $b$,则 $S \subseteq [a, b]$,其中 $a \in S$。

对 $[a, b]$ 二等分为 $\left[ a, \frac{a + b}{2} \right]$ 和 $\left[ \frac{a + b}{2}, b \right]$,若 $\frac{a + b}{2}$ 非 $S$ 的上界,则记 $[a_1, b_1] = \left[ a, \frac{a + b}{2} \right]$,否则记 $[a_1, b_1] = \left[ \frac{a + b}{2}, b \right]$。

对 $[a_1, b_1]$ 继续二等分为 $\left[ a_1, \frac{a_1 + b_1}{2} \right]$ 和 $\left[ \frac{a_1 + b_1}{2}, b_1 \right]$,若 $\frac{a_1 + b_1}{2}$ 非 $S$ 的上界,则记 $[a_2, b_2] = \left[ a_1, \frac{a_1 + b_1}{2} \right]$,否则记 $[a_2, b_2] = \left[ \frac{a_1 + b_1}{2}, b_1 \right]$。

如此递推,对 $[a_{n-1}, b_{n-1}]$ 二等分为 $\left[ a_{n-1}, \frac{a_{n-1} + b_{n-1}}{2} \right]$ 和 $\left[ \frac{a_{n-1} + b_{n-1}}{2}, b_{n-1} \right]$,若 $\frac{a_{n-1} + b_{n-1}}{2}$ 非 $S$ 的上界,则记 $[a_n, b_n] = \left[ a_{n-1}, \frac{a_{n-1} + b_{n-1}}{2} \right]$,否则记 $[a_n, b_n] = \left[ \frac{a_{n-1} + b_{n-1}}{2}, b_{n-1} \right]$。

由此,得到一列闭区间 $\{[a_n, b_n]\}$。

易知数列 $\{a_n\}$ 单调增加有上界,数列 $\{b_n\}$ 单调减少有下界且 $\forall x \in S$ 有 $b \geq x$。根据单调有界定理,可知 $\{a_n\}$ 与 $\{b_n\}$ 均收敛,存在 $\xi \in \mathbb{R}$,使得 $\lim_{n \to \infty} b_n =  \xi $,又
\[
|a_n - b_n| = \left( \frac{1}{2} \right)^n (b - a),
\]
因此 $\lim_{n \to \infty} a_n = \lim_{n \to \infty} b_n =  \xi $。

下说明 $\xi $ 是 $S$ 的上确界:
\begin{enumerate}
    \item $\forall x \in S$,有 $\xi  \geq x$;
    \item $\forall \varepsilon > 0$,$\exists N \in \mathbb{N}^{+}$,使得 $\xi - a_n < \varepsilon$,即 $\xi < a_n + \varepsilon$,其中 $a_n \in S$。
\end{enumerate}

综上所述,$\xi $ 为 $S$ 的上确界。证毕。


\section*{单调有界定理 $\Rightarrow$ Cauchy 收敛原理}

\textbf{证明:}(必要条件) 先证数列 $\{x_n\}$ 收敛 $\Rightarrow$ 数列 $\{x_n\}$ 是基本数列。\\
设 $\lim_{n \to \infty} x_n = a$,则 $\forall \varepsilon > 0$,$\exists N \in \mathbb{N}^+$,使得
\[
\forall n, m > N, \, 都有 \, |x_n - a| < \frac{\varepsilon}{2}, \, |x_m - a| < \frac{\varepsilon}{2},
\]
从而
\[
|x_n - x_m| \leq |x_n - a| + |x_m - a| < \varepsilon,
\]
因此数列 $\{x_n\}$ 是基本数列。

再证(充分条件)数列 $\{x_n\}$ 是基本数列 $\Rightarrow$ 数列 $\{x_n\}$ 收敛。\\
设数列 $\{x_n\}$ 是基本数列,下说明数列 $\{x_n\}$ 有界。$\forall \varepsilon > 0$,$\exists N \in \mathbb{N}^+$,$\forall n, m > N$,都有 $|x_n - x_m| < \varepsilon$。取定 $\varepsilon = 1$ 时,存在 $N_1 \in \mathbb{N}^+$,使得
\[
\forall n > N_1, \, |x_n - x_{N_1}| < 1,
\]
于是 $M = \max \{x_1, x_2, \ldots, x_{N_1}, |x_{N_1}| + 1\}$ 是数列 $\{x_n\}$ 的一个上界,
\[
m = \min \{x_1, x_2, \ldots, x_{N_1}, -1\}
\]
是数列 $\{x_n\}$ 的一个下界,因此数列 $\{x_n\}$ 有界。下说明数列 $\{x_n\}$ 收敛。将 $[m, M]$ 均分为
\[
\left[m, \frac{m + M}{2}\right], \left[\frac{m + M}{2}, M\right],
\]
则必有其中之一闭区间含有 $\{x_n\}$ 中无穷个点,记此闭区间为 $[m_1, M_1]$。将 $[m_1, M_1]$ 均分为
\[
\left[m_1, \frac{m_1 + M_1}{2}\right], \left[\frac{m_1 + M_1}{2}, M_1\right],
\]
则必有其中之一闭区间含有 $\{x_n\}$ 中无穷个点,记此闭区间为 $[m_2, M_2]$。如此一直下去,得到一列闭区间 $\{[m_n, M_n]\}$,且
\[
\lim_{n \to \infty} (M_n - m_n) = \lim_{n \to \infty} \frac{1}{2^n} (M - m) = 0,
\]
易知数列 $\{m_n\}$ 单调增加,数列 $\{M_n\}$ 单调减少,根据 \textcolor{red}{单调有界定理},存在实数 $\alpha, \beta$ 使得
\[
\lim_{n \to \infty} m_n = \alpha, \quad \lim_{n \to \infty} M_n = \beta,
\]
且
\[
m_n \leq \alpha \leq \beta \leq M_n, \quad n = 1, 2, 3, \ldots
\]
可知 $\lim_{n \to \infty} m_n = \alpha = \beta = \lim_{n \to \infty} M_n$。\\
令每个 $[m_n, M_n]$ 中取一个 $\{x_n\}$ 中的元素记为 $x_{n_k}$,其中 $x_n$ 与 $x_{n_{k-1}}$ 相异,易知 $\lim_{k \to \infty} x_{n_k} = \alpha$。又数列 $\{x_n\}$ 是基本数列,
\[
\forall \varepsilon > 0,\exists N \in \mathbb{N}^+, \forall n, m > N, \, |x_n - x_m| < \frac{\varepsilon}{2}。
\]
而根据 $\lim_{k \to \infty} x_{n_k} = \alpha$ 有
\[
\forall \varepsilon > 0, \exists K \in \mathbb{N}^+, n_k > N,k > K,\text{有} \, |x_{n_k} - \alpha| < \frac{\varepsilon}{2}。
\]
于是 $\forall \varepsilon > 0,\forall n > N,k > K$ 有
\[
|x_n - \alpha| \leq |x_n - x_{n_k}| + |x_{n_k} - \alpha| < \varepsilon,
\]
因此数列 $\{x_n\}$ 收敛,证毕。


\section*{单调有界定理 $\Rightarrow$ 闭区间套定理}

\textbf{证明:} 设一列闭区间 $\{[a_n, b_n]\}$ 形成一个闭区间套,则数列 $\{a_n\}$ 单调递增有上界,数列 $\{b_n\}$ 单调递减有下界,\textbf{存在性},根据 \textcolor{red}{单调有界定理},存在实数 $\alpha, \beta$ 使得 $\lim_{n \to \infty} a_n = \alpha$,$\lim_{n \to \infty} b_n = \beta$,又 $\lim_{n \to \infty} (b_n - a_n) = 0$,又有
\[
a_n \leq \alpha \leq \beta \leq b_n,\quad n = 1, 2, 3, \ldots
\]
可知 $\lim_{n \to \infty} a_n = \alpha = \beta = \lim_{n \to \infty} b_n$。\textbf{唯一性},若另有 $\alpha'$,使得 $a_n \leq \alpha' \leq b_n,\quad n = 1, 2, 3, \ldots$ 且 $\lim_{n \to \infty} a_n = \lim_{n \to \infty} b_n = \alpha'$,由 $\lim_{n \to \infty} (b_n - a_n) = 0$,可知 $\alpha = \alpha'$,证毕。




\section*{单调有界定理 $\Rightarrow$ 聚点定理}

\subsection*{方法一}
\textbf{证明:} 设 $S$ 是有界无穷点集,取存在闭区间 $[a, b]$,使得 $S \subseteq [a, b]$。将 $[a, b]$ 均分为
\[
\left[ a, \frac{a + b}{2} \right], \left[ \frac{a + b}{2}, b \right],
\]
则必在其中之一闭区间含有 $S$ 中无穷个点,记此闭区间为
\[
\left[ a_1, b_1 \right] = \left[ a, \frac{a + b}{2} \right] \text{ 或 } \left[ \frac{a + b}{2}, b \right]。
\]
将 $[a_1, b_1]$ 均分为
\[
\left[ a_1, \frac{a_1 + b_1}{2} \right], \left[ \frac{a_1 + b_1}{2}, b_1 \right],
\]
则必在其中之一闭区间含有 $S$ 中无穷个点,记此闭区间为
\[
\left[ a_2, b_2 \right] = \left[ a_1, \frac{a_1 + b_1}{2} \right] \text{ 或 } \left[ \frac{a_1 + b_1}{2}, b_1 \right]。
\]
如此一直下去,得到一列闭区间 $\{[a_n, b_n]\}$,且 $\lim_{n \to \infty} (b_n - a_n) = \lim_{n \to \infty} \frac{1}{2^n} (b - a) = 0$,易知数列 $\{a_n\}$ 单调增加,数列 $\{b_n\}$ 单调减少,根据 \textcolor{red}{单调有界定理},存在实数 $\alpha, \beta$ 使得
\[
\lim_{n \to \infty} a_n = \alpha, \quad \lim_{n \to \infty} b_n = \beta,
\]
且
\[
a_n \leq \alpha \leq \beta \leq b_n, \quad n = 1, 2, 3, \ldots
\]
可知
\[
\lim_{n \to \infty} a_n = \alpha = \beta = \lim_{n \to \infty} b_n。
\]
下说明 $\alpha$ 是 $S$ 的一个聚点。在每个 $[a_n, b_n]$ 中取一个 $S$ 中的元素记为 $x_n$,易知 $\lim_{n \to \infty} x_n = \alpha$,因此 $\alpha$ 是 $S$ 的聚点,证毕。

\subsection*{方法二(证明致密性定理)}
\textbf{证明:} 先说明任意有界数列必有单调子列,设 $\{x_n\}$ 是一个有界数列,定义数列 $\{x_n\}$ 中的第 $k$ 项 $x_k$ 具有性质 $A$ 是指:$x_k = \max \{x_i | i \geq k\}$。

情形一,数列 $\{x_n\}$ 中有无限项具有性质 $A$,不妨设为 $x_{n_1}, x_{n_2}, \ldots, x_{n_k}, \ldots$,其中
\[
n_1 < n_2 < \cdots < n_k < \cdots,
\]
则数列 $\{x_{n_k}\}$ 单调减少且有界;

情形二,数列 $\{x_n\}$ 中只有有限项具有性质 $A$,则存在 $N \in \mathbb{N}^+$,$\forall n > N$,$x_n$ 都不具有性质 $A$。任取一项记作 $x_{n_1}$,其中 $n_1 > N$,由于 $x_{n_1}$ 不具有性质 $A$,则必存在 $n_2 > n_1$ 使得 $x_{n_2} > x_{n_1}$,由于 $x_{n_2}$ 不具有性质 $A$,则必存在 $n_3 > n_2$ 使得 $x_{n_3} > x_{n_2}$,如此继续下去可以得到一个单调增加的数列 $\{x_{n_k}\}$。

从而证明了任意有界数列必有单调子列。根据 \textcolor{red}{单调有界定理},数列 $\{x_{n_k}\}$ 必收敛,于是对于任意无界数集 $S$,必有聚点。

\section*{单调有界定理 $\Rightarrow$ 有限覆盖定理}

\textbf{证明:} 设闭区间 $[a, b]$ 在实数 $R$ 上,任取闭区间 $[a, b]$ 的一个开覆盖 $\{O_{\lambda}\}$,反证法,假设闭区间 $[a, b]$ 不能被 $\{O_{\lambda}\}$ 的有限个子集覆盖。将 $[a, b]$ 二等分为
\[
\left[ a, \frac{a + b}{2} \right], \left[ \frac{a + b}{2}, b \right],
\]
至少有其中之一不能被 $\{O_{\lambda}\}$ 的有限个子集覆盖,将此区间记为 $[a_1, b_1]$。将 $[a_1, b_1]$ 二等分为
\[
\left[ a_1, \frac{a_1 + b_1}{2} \right], \left[ \frac{a_1 + b_1}{2}, b_1 \right],
\]
至少有其中之一不能被 $\{O_{\lambda}\}$ 的有限个子集覆盖,将此区间记为 $[a_2, b_2]$。如此下去,便可得到一列闭区间 $\{[a_n, b_n]\}$,且 $\lim_{n \to \infty} (b_n - a_n) = \lim_{n \to \infty} \frac{1}{2^n} (b - a) = 0$。易知数列 $\{a_n\}$ 单调增加,数列 $\{b_n\}$ 单调减少,根据 \textcolor{red}{单调有界定理},数列 $\{a_n\}$,$\{b_n\}$ 收敛,又
\[
\lim_{n \to \infty} (b_n - a_n) = 0,
\]
可知存在 $\xi$,使得
\[
\lim_{n \to \infty} a_n = \lim_{n \to \infty} b_n = \xi,\quad a_n \leq \xi \leq b_n,\quad n = 1, 2, \ldots。
\]

由于存在 $\{O_{\lambda}\}$ 的一个开区间 $O_{\lambda^*}$,使得 $\xi \in O_{\lambda^*}$,当 $n$ 充分大时,必有 $[a_n, b_n] \subseteq O_{\lambda^*}$ 与 $[a, b]$ 不能被 $\{O_{\lambda}\}$ 的有限个子集覆盖矛盾,因此存在 $\{O_{\lambda}\}$ 的有限个子集覆盖闭区间 $[a, b]$,证毕。


\begin{theorem}[Cauchy 收敛准则] \label{pro:max3}
(1)数列 \(\{x_n\}\) 收敛的充要条件是 \(\{x_n\}\) 是柯西数列。

(2)一个实数数列 $\{a_n\}$ 收敛,当且仅当对任意的 $\varepsilon > 0$,存在一个正整数 $N$,使得当 $n, m > N$ 时,满足
\[
|a_n - a_m| < \varepsilon。
\]
那么数列 $\{a_n\}$ 是收敛的。
\end{theorem}

\section*{Cauchy 收敛准则 $\Rightarrow$ 确界存在定理}

\textbf{证明:} 设 $S$ 是非空有上界数集,任取它的一个上界 $b \notin S$,任取 $a \in S$,则 $S \subseteq [a, b]$。将 $[a, b]$ 均分为
\[
\left[ a, \frac{a + b}{2} \right], \left[ \frac{a + b}{2}, b \right],
\]
若 $\frac{a + b}{2}$ 非 $S$ 的上界,则记 $[a_1, b_1] = \left[ a, \frac{a + b}{2} \right]$,否则记 $[a_1, b_1] = \left[ \frac{a + b}{2}, b \right]$。对 $[a_1, b_1]$ 均分为
\[
\left[ a_1, \frac{a_1 + b_1}{2} \right], \left[ \frac{a_1 + b_1}{2}, b_1 \right],
\]
若 $\frac{a_1 + b_1}{2}$ 非 $S$ 的上界,则记 $[a_2, b_2] = \left[ a_1, \frac{a_1 + b_1}{2} \right]$,否则记 $[a_2, b_2] = \left[ \frac{a_1 + b_1}{2}, b_1 \right]$。依此类推,对 $[a_{n-1}, b_{n-1}]$ 均分为
\[
\left[ a_{n-1}, \frac{a_{n-1} + b_{n-1}}{2} \right], \left[ \frac{a_{n-1} + b_{n-1}}{2}, b_{n-1} \right],
\]
若 $\frac{a_{n-1} + b_{n-1}}{2}$ 非 $S$ 的上界,则记 $[a_n, b_n] = \left[ a_{n-1}, \frac{a_{n-1} + b_{n-1}}{2} \right]$,否则记 $[a_n, b_n] = \left[ \frac{a_{n-1} + b_{n-1}}{2}, b_{n-1} \right]$。依此类推,得到一列闭区间 $\{[a_n, b_n]\}$。易知 $\{[a_n, b_n]\}$ 满足:
\begin{enumerate}
    \item $\{[a_n, b_n]\} \subseteq [a, b]$,且 $\{[a_{n+1}, b_{n+1}]\} \subseteq [a_n, b_n]$,$n = 1, 2, 3, \ldots$;
    \item $\lim_{n \to \infty} (b_n - a_n) = \lim_{n \to \infty} \left(\frac{1}{2}\right)^n (b - a) = 0$。
\end{enumerate}

因此,$\{[a_n, b_n]\}$ 形成一个闭区间套。$\forall \varepsilon > 0$,$\exists N \in \mathbb{N}^+$,使得 $\left(\frac{1}{2}\right)^N (b - a) < \varepsilon$,则 $\forall n, m > N$,有
\[
|a_n - a_m| < |a_n - b_n| = \left(\frac{1}{2}\right)^n (b - a) < \varepsilon,
\]
\[
|b_n - b_m| < |b_n - a_n| = \left(\frac{1}{2}\right)^n (b - a)  < \varepsilon,
\]
因此数列 $\{a_n\}$,$\{b_n\}$ 均是基本数列。根据 \textcolor{red}{Cauchy 收敛准则},$\{a_n\}$,$\{b_n\}$ 均收敛,设 $\lim_{n \to \infty} a_n = \lim_{n \to \infty} b_n = \xi$,下说明 $\xi$ 是 $S$ 的上确界。\\
$\forall x \in S$,有 $x \leq \xi$;$\forall \varepsilon > 0$,$\exists N \in \mathbb{N}^+$,有$n> N$,使得 $a_n \leq x < \xi + \varepsilon$,即 $b_n \leq \xi + \varepsilon$,从而 $b_n \in S$。综上,$\xi$ 是 $S$ 的上确界,证毕。


\section*{Cauchy 收敛准则 $\Rightarrow$ 单调有界定理}

\textbf{证明:} 设数列 $\{x_n\}$ 是单调增加有上界,令 $S = \{x_1, x_2, \ldots\}$,任取它的一个上界 $b \notin S$,任取 $a \in S$,则 $S \subseteq [a, b]$。对 $[a, b]$ 二等分为
\[
\left[ a, \frac{a + b}{2} \right], \left[ \frac{a + b}{2}, b \right],
\]
若 $\frac{a + b}{2}$ 非 $S$ 的上界,则记 $[a_1, b_1] = \left[ a, \frac{a + b}{2} \right]$,否则记 $[a_1, b_1] = \left[ \frac{a + b}{2}, b \right]$。对 $[a_1, b_1]$ 二等分为
\[
\left[ a_1, \frac{a_1 + b_1}{2} \right], \left[ \frac{a_1 + b_1}{2}, b_1 \right],
\]
若 $\frac{a_1 + b_1}{2}$ 非 $S$ 的上界,则记 $[a_2, b_2] = \left[ a_1, \frac{a_1 + b_1}{2} \right]$,否则记 $[a_2, b_2] = \left[ \frac{a_1 + b_1}{2}, b_1 \right]$。依此类推,对 $[a_{n-1}, b_{n-1}]$ 二等分为
\[
\left[ a_{n-1}, \frac{a_{n-1} + b_{n-1}}{2} \right], \left[ \frac{a_{n-1} + b_{n-1}}{2}, b_{n-1} \right],
\]
若 $\frac{a_{n-1} + b_{n-1}}{2}$ 非 $S$ 的上界,则记 $[a_n, b_n] = \left[ a_{n-1}, \frac{a_{n-1} + b_{n-1}}{2} \right]$,否则记 $[a_n, b_n] = \left[ \frac{a_{n-1} + b_{n-1}}{2}, b_{n-1} \right]$。依此类推,得到一列闭区间 $\{[a_n, b_n]\}$。$\forall \varepsilon > 0$,由
\[
b_n - a_n = \frac{1}{2^n} (b - a),
\]
可知 $\exists N \in \mathbb{N}^+$,$n>N$,使得 $\frac{1}{2^n} (b - a) < \varepsilon$,于是 $\forall n, m > N$,有
\[
|a_n - a_m| < |b_n - a_n| < \varepsilon, \quad |b_n - b_m| < |b_n - a_n| < \varepsilon,
\]
因此数列 $\{a_n\}$,$\{b_n\}$ 是 Cauchy 数列,根据 \textcolor{red}{Cauchy 收敛准则},数列 $\{a_n\}$,$\{b_n\}$ 都收敛,且由 $\lim_{n \to \infty} (b_n - a_n) = \frac{1}{2^n} (b - a) = 0$ 知,
\[
\lim_{n \to \infty} a_n = \lim_{n \to \infty} b_n = \xi。
\]

在$[a_n, b_n]$ 中取某个 $x_n^k \in \{x_1, x_2, \ldots\}$,于是有 $\{x_n^k\}$ 的子列收敛到 $\xi$,即 $\lim_{k \to \infty} x_n^k = \xi$,易知 $\lim_{n \to \infty} x_n = \xi$



\section*{Cauchy 收敛准则 $\Rightarrow$ 闭区间套定理}

\textbf{证明:} 设一列闭区间 $\{[a_n, b_n]\}$ 形成一个闭区间套,存在性,由 $\lim_{n \to \infty} (b_n - a_n) = 0$ 知,$\forall \varepsilon > 0$,$\exists N \in \mathbb{N}^+$,使得 $b_n - a_n < \varepsilon$,于是 $\forall n, m > N$,有 $|a_n - a_m| < \varepsilon$,$|b_n - b_m| < \varepsilon$。因此数列 $\{a_n\}$,$\{b_n\}$ 是 Cauchy 数列,根据 \textcolor{red}{Cauchy 收敛准则},数列 $\{a_n\}$,$\{b_n\}$ 都收敛。由 $\lim_{n \to \infty} (b_n - a_n) = 0$ 知
\[
\lim_{n \to \infty} a_n = \lim_{n \to \infty} b_n = \xi,
\]
并设收敛到 $\xi$。唯一性,若另有 $\xi'$,使得 $a_n \leq \xi' \leq b_n,n = 1, 2, 3, \ldots$,且 $\lim_{n \to \infty} a_n = \lim_{n \to \infty} b_n = \xi'$,由 $\lim_{n \to \infty} (b_n - a_n) = 0$,可知 $\xi = \xi'$,证毕。


\section*{Cauchy 收敛准则 $\Rightarrow$ 聚点定理}

\textbf{证明:} 设 $S$ 是有界无穷点集,则存在闭区间 $[a, b]$,使得 $S \subseteq [a, b]$。将 $[a, b]$ 均分为
\[
\left[ a, \frac{a + b}{2} \right], \left[ \frac{a + b}{2}, b \right],
\]
则必有其中之一闭区间含有 $S$ 中无穷个点,记此闭区间为 $[a_1, b_1]$。将 $[a_1, b_1]$ 均分为
\[
\left[ a_1, \frac{a_1 + b_1}{2} \right], \left[ \frac{a_1 + b_1}{2}, b_1 \right],
\]
则必有其中之一闭区间含有 $S$ 中无穷个点,记此闭区间为 $[a_2, b_2]$。$\cdots$,将 $[a_{k-1}, b_{k-1}]$ 均分为
\[
\left[ a_{k-1}, \frac{a_{k-1} + b_{k-1}}{2} \right], \left[ \frac{a_{k-1} + b_{k-1}}{2}, b_{k-1} \right],
\]
则必有其中之一闭区间含有 $S$ 中无穷个点,记此闭区间为 $[a_k, b_k]$。如此一直下去,得到一列闭区间 $\{[a_n, b_n]\}$,且
\[
\lim_{n \to \infty} (b_n - a_n) = \lim_{n \to \infty} \frac{1}{2^n} (b - a) = 0,
\]
易知 $\{a_n\}$ 单调增加,数列 $\{b_n\}$ 单调减少,且 $\forall \varepsilon > 0$,$\exists N \in \mathbb{N}^+$,使得 $b_n - a_n < \varepsilon$,于是 $\forall n, m > N$,有
\[
|a_n - a_m| < |b_n - a_n| < \varepsilon,
\]
\[
|b_n - b_m| < |b_n - a_n| < \varepsilon,
\]
因此数列 $\{a_n\}$,$\{b_n\}$ 是 Cauchy 数列,根据 \textcolor{red}{Cauchy 收敛准则},数列 $\{a_n\}$,$\{b_n\}$ 都收敛,设收敛到 $\xi$。由 $\lim_{n \to \infty} (b_n - a_n) = 0$ 知
\[
\lim_{n \to \infty} a_n = \lim_{n \to \infty} b_n = \xi,
\]
下说明 $\xi$ 是 $S$ 的一个聚点。在每个 $[a_n, b_n]$ 中取一个 $S$ 中的元素记为 $x_n$,易知 $\lim_{n \to \infty} x_n = \xi$,因此 $\alpha = \xi$ 是 $S$ 的聚点,证毕。

\section*{Cauchy 收敛准则 $\Rightarrow$ 有限覆盖定理}

\textbf{证明:} 设闭区间 $[a, b]$ 在实数 $R$ 上,任取闭区间 $[a, b]$ 的一个开覆盖 $\{O_{\lambda}\}$,反证法,假设闭区间 $[a, b]$ 不能被 $\{O_{\lambda}\}$ 的有限个子集覆盖,将 $[a, b]$ 二等分为
\[
\left[ a, \frac{a + b}{2} \right], \left[ \frac{a + b}{2}, b \right],
\]
至少有其中之一不能被 $\{O_{\lambda}\}$ 的有限个子集覆盖,将此区间记为 $[a_1, b_1]$。将 $[a_1, b_1]$ 二等分为
\[
\left[ a_1, \frac{a_1 + b_1}{2} \right], \left[ \frac{a_1 + b_1}{2}, b_1 \right],
\]
至少有其中之一不能被 $\{O_{\lambda}\}$ 的有限个子集覆盖,将此区间记为 $[a_2, b_2]$。如此下去,便可得到一列闭区间 $\{[a_n, b_n]\}$,且
\[
\lim_{n \to \infty} (b_n - a_n) = \lim_{n \to \infty} \left(\frac{1}{2}\right)^n (b - a) = 0。
\]

易知数列 $\{a_n\}$ 单调增加,数列 $\{b_n\}$ 单调减少,且 $\forall \varepsilon > 0$,$\exists N \in \mathbb{N}^+$,使得 $b_n - a_n < \varepsilon$,于是 $\forall n, m > N$,有
\[
|a_n - a_m| < |b_n - a_n| < \varepsilon,
\]
\[
|b_n - b_m| < |b_n - a_n| < \varepsilon,
\]
因此数列 $\{a_n\}$,$\{b_n\}$ 是 Cauchy 数列,根据 \textcolor{red}{Cauchy 收敛准则},数列 $\{a_n\}$,$\{b_n\}$ 都收敛,设收敛到 $\xi$。由 $\lim_{n \to \infty} (b_n - a_n) = 0$ 知
\[
\lim_{n \to \infty} a_n = \lim_{n \to \infty} b_n = \xi,
\]
且 $\xi \in [a, b]$,于是存在 $\{O_{\lambda}\}$ 的一个开区间 $O_{\lambda^*}$,使得 $\xi \in O_{\lambda^*}$,当 $n$ 充分大时,必有 $[a_n, b_n] \subseteq O_{\lambda^*}$ 与 $[a, b]$ 不能被 $\{O_{\lambda}\}$ 的有限个子集覆盖矛盾。因此存在 $\{O_{\lambda}\}$ 的有限个子集覆盖闭区间 $[a, b]$,证毕。



\begin{theorem}[ 闭区间套定理] \label{property:cauchy3}
(1)若 \(\{[a_n, b_n]\}\) 形成一个闭区间套,则存在唯一的实数 \(\xi\) 属于所有的闭区间 \([a_n, b_n]\),\(n = 1, 2, 3, \cdots\),且存在唯一的实数 \( \xi \),使得 \( \xi \) 属于所有这些闭区间,即  \( \lim_{n \to \infty} b_n = 0 \) = \( \lim_{n \to \infty} a_n   \) =\( \xi \) 

(2)\begin{enumerate}
    \item $[a_{n+1}, b_{n+1}] \subseteq [a_n, b_n]$ 对于所有正整数 $n$ 都成立;
    \item $\lim_{n \to \infty} (b_n - a_n) = 0$。
\end{enumerate}
则存在唯一的实数 $\xi$,使得
\[
\bigcap_{n=1}^{\infty} [a_n, b_n] = \{\xi\}。
\]
\end{theorem}


\section*{闭区间套定理 $\Rightarrow$ 确界存在定理}

\textbf{证明:} 设 $S$ 是非空有上界数集,任取它的一个上界 $b \notin S$,任取 $a \in S$,则 $S \subseteq [a, b]$。对 $[a, b]$ 二等分为
\[
\left[ a, \frac{a + b}{2} \right], \left[ \frac{a + b}{2}, b \right],
\]
若 $\frac{a + b}{2}$ 非 $S$ 的上界,则记 $[a_1, b_1] = \left[ a, \frac{a + b}{2} \right]$,否则记 $[a_1, b_1] = \left[ \frac{a + b}{2}, b \right]$。对 $[a_1, b_1]$ 二等分为
\[
\left[ a_1, \frac{a_1 + b_1}{2} \right], \left[ \frac{a_1 + b_1}{2}, b_1 \right],
\]
若 $\frac{a_1 + b_1}{2}$ 非 $S$ 的上界,则记 $[a_2, b_2] = \left[ a_1, \frac{a_1 + b_1}{2} \right]$,否则记 $[a_2, b_2] = \left[ \frac{a_1 + b_1}{2}, b_1 \right]$。依此类推,对 $[a_{n-1}, b_{n-1}]$ 二等分为
\[
\left[ a_{n-1}, \frac{a_{n-1} + b_{n-1}}{2} \right], \left[ \frac{a_{n-1} + b_{n-1}}{2}, b_{n-1} \right],
\]
若 $\frac{a_{n-1} + b_{n-1}}{2}$ 非 $S$ 的上界,则记 $[a_n, b_n] = \left[ a_{n-1}, \frac{a_{n-1} + b_{n-1}}{2} \right]$,否则记 $[a_n, b_n] = \left[ \frac{a_{n-1} + b_{n-1}}{2}, b_{n-1} \right]$。依此类推,得到一列闭区间 $\{[a_n, b_n]\}$。易知 $\{[a_n, b_n]\}$ 满足:
\begin{enumerate}
    \item $\{[a_{n+1}, b_{n+1}] \subseteq [a_n, b_n]\}$ 对于所有正整数 $n$ 都成立;
    \item $\lim_{n \to \infty} (b_n - a_n) = \lim_{n \to \infty} \left(\frac{1}{2}\right)^n (b - a) = 0$。
\end{enumerate}

因此,$\{[a_n, b_n]\}$ 形成一个闭区间套。根据 \textcolor{red}{闭区间套定理},存在实数 $\xi \in \mathbb{R}$,使得
\[
a_n \leq \xi \leq b_n,\quad n = 1, 2, 3, \ldots,
\]
且
\[
\lim_{n \to \infty} a_n = \lim_{n \to \infty} b_n = \xi。
\]

下说明 $\xi$ 是 $S$ 的上确界:
\begin{enumerate}
    \item $\forall x \in S$,有 $x \leq \xi$;
    \item $\forall \varepsilon > 0$,$\exists N \in \mathbb{N}^+$,使得 $\xi - a_n < \varepsilon$,即 $b < \xi + \varepsilon$,其中 $a_n \in S$。
\end{enumerate}

因此,$\xi$ 是 $S$ 的上确界,证毕。


\section*{闭区间套定理 $\Rightarrow$ 单调有界定理}

\textbf{证明:} 设数列 $\{x_n\}$ 是单调增加有上界,令 $S = \{x_1, x_2, \ldots\}$,任取它的一个上界 $b \notin S$,任取 $a \in S$,则 $S \subseteq [a, b]$。对 $[a, b]$ 二等分为
\[
\left[ a, \frac{a + b}{2} \right], \left[ \frac{a + b}{2}, b \right],
\]
若 $\frac{a + b}{2}$ 非 $S$ 的上界,则记 $[a_1, b_1] = \left[ a, \frac{a + b}{2} \right]$,否则记 $[a_1, b_1] = \left[ \frac{a + b}{2}, b \right]$。对 $[a_1, b_1]$ 二等分为
\[
\left[ a_1, \frac{a_1 + b_1}{2} \right], \left[ \frac{a_1 + b_1}{2}, b_1 \right],
\]
若 $\frac{a_1 + b_1}{2}$ 非 $S$ 的上界,则记 $[a_2, b_2] = \left[ a_1, \frac{a_1 + b_1}{2} \right]$,否则记 $[a_2, b_2] = \left[ \frac{a_1 + b_1}{2}, b_1 \right]$。依此类推,对 $[a_{n-1}, b_{n-1}]$ 二等分为
\[
\left[ a_{n-1}, \frac{a_{n-1} + b_{n-1}}{2} \right], \left[ \frac{a_{n-1} + b_{n-1}}{2}, b_{n-1} \right],
\]
若 $\frac{a_{n-1} + b_{n-1}}{2}$ 非 $S$ 的上界,则记 $[a_n, b_n] = \left[ a_{n-1}, \frac{a_{n-1} + b_{n-1}}{2} \right]$,否则记 $[a_n, b_n] = \left[ \frac{a_{n-1} + b_{n-1}}{2}, b_{n-1} \right]$。依此类推,得到一列闭区间 $\{[a_n, b_n]\}$。易知 $\{[a_n, b_n]\}$ 满足:
\begin{enumerate}
    \item $\{[a_{n+1}, b_{n+1}] \subseteq [a_n, b_n]\}$ 对于所有正整数 $n$ 都成立;
    \item $\lim_{n \to \infty} (b_n - a_n) = \lim_{n \to \infty} \left(\frac{1}{2}\right)^n (b - a) = 0$。
\end{enumerate}

因此,$\{[a_n, b_n]\}$ 形成一个闭区间套。根据 \textcolor{red}{闭区间套定理},存在唯一的实数 $\xi \in \mathbb{R}$,使得
\[
a_n \leq \xi \leq b_n,\quad n = 1, 2, 3, \ldots,
\]
且
\[
\lim_{n \to \infty} a_n = \lim_{n \to \infty} b_n = \xi。
\]

下说明 $\xi$ 是 $S$ 的上确界。在每个 $[a_n, b_n]$ 中取某个 $x_n^k \in \{x_1, x_2, \ldots\}$,于是有 $\{x_n^k\}$ 的子列收敛到 $\xi$,即 $\lim_{k \to \infty} x_n^k = \xi$,易知 $\lim_{n \to \infty} x_n = \xi$,因此 $S$ 的上确界是 $\xi$,证毕。

\section*{闭区间套定理 $\Rightarrow$ Cauchy 收敛准则}

\textbf{证明:} 先证数列 $\{x_n\}$ 收敛 $\Rightarrow$ 数列 $\{x_n\}$ 是基本数列。设 $\lim_{n \to \infty} x_n = a$,则 $\forall \varepsilon > 0$,$\exists N \in \mathbb{N}^+$,使得
\[
\forall n, m > N,\, 都有 \, |x_n - a| < \frac{\varepsilon}{2},\, |x_m - a| < \frac{\varepsilon}{2},
\]
从而
\[
|x_n - x_m| \leq |x_n - a| + |x_m - a| < \varepsilon,
\]
因此数列 $\{x_n\}$ 是基本数列。

再证数列 $\{x_n\}$ 是基本数列 $\Rightarrow$ 数列 $\{x_n\}$ 收敛。设数列 $\{x_n\}$ 是基本数列,下说明数列 $\{x_n\}$ 有界。$\forall \varepsilon > 0$,$\exists N \in \mathbb{N}^+$,$\forall n, m > N$,都有 $|x_n - x_m| < \varepsilon$。取定 $\varepsilon = 1$ 时,存在 $N_1 \in \mathbb{N}^+$,使得
\[
\forall n > N_1, \, |x_n - x_{N_1}| < \varepsilon,
\]
于是 $M = \max \{x_1, x_2, \ldots, x_{N_1}, |x_{N_1}| + 1\}$ 是数列 $\{x_n\}$ 的一个上界,
\[
m = \min \{x_1, x_2, \ldots, x_{N_1}, -1\}
\]
是数列 $\{x_n\}$ 的一个下界,因此数列 $\{x_n\}$ 有界。下说明数列 $\{x_n\}$ 收敛。将 $[m, M]$ 均分为
\[
\left[m, \frac{m + M}{2}\right], \left[\frac{m + M}{2}, M\right],
\]
则必有其中之一闭区间含有 $\{x_n\}$ 中无穷个点,记此闭区间为 $[m_1, M_1]$。将 $[m_1, M_1]$ 均分为
\[
\left[m_1, \frac{m_1 + M_1}{2}\right], \left[\frac{m_1 + M_1}{2}, M_1\right],
\]
则必有其中之一闭区间含有 $\{x_n\}$ 中无穷个点,记此闭区间为 $[m_2, M_2]$。$\cdots$,将 $[m_{k-1}, M_{k-1}]$ 均分为
\[
\left[m_{k-1}, \frac{m_{k-1} + M_{k-1}}{2}\right], \left[\frac{m_{k-1} + M_{k-1}}{2}, M_{k-1}\right],
\]
则必有其中之一闭区间含有 $\{x_n\}$ 中无穷个点,记此闭区间为 $[m_k, M_k]$。如此一直下去,得到一列闭区间 $\{[m_n, M_n]\}$,且
\[
\lim_{n \to \infty} (M_n - m_n) = \lim_{n \to \infty} \frac{1}{2^n} (M - m) = 0,
\]
易知 $\{m_n\}$ 单调增加,数列 $\{M_n\}$ 单调减少,根据 \textcolor{red}{闭区间套定理},存在实数 $\alpha, \beta$ 使得
\[
\lim_{n \to \infty} m_n = \alpha, \quad \lim_{n \to \infty} M_n = \beta,
\]
且
\[
m_n \leq \alpha \leq \beta \leq M_n, \quad n = 1, 2, 3, \ldots
\]
可知 $\lim_{n \to \infty} m_n = \alpha = \beta = \lim_{n \to \infty} M_n$,即 $\lim_{n \to \infty} m_n = \lim_{n \to \infty} M_n = \xi$。下说明 $\xi$ 是 $S$ 的一个聚点。在每个 $[m_n, M_n]$ 中取一个 $S$ 中的元素记为 $x_n$,易知 $\lim_{n \to \infty} x_n = \xi$。因此数列 $\{x_n\}$ 收敛,证毕。


\section*{闭区间套定理 $\Rightarrow$ 聚点定理}

\textbf{证明:} 设 $S$ 是有界无穷点集,则存在闭区间 $[a, b]$,使得 $S \subseteq [a, b]$。将 $[a, b]$ 均分为
\[
\left[ a, \frac{a + b}{2} \right], \left[ \frac{a + b}{2}, b \right],
\]
则必有其中之一闭区间含有 $S$ 中无穷个点,记此闭区间为 $[a_1, b_1]$。将 $[a_1, b_1]$ 均分为
\[
\left[ a_1, \frac{a_1 + b_1}{2} \right], \left[ \frac{a_1 + b_1}{2}, b_1 \right],
\]
则必有其中之一闭区间含有 $S$ 中无穷个点,记此闭区间为 $[a_2, b_2]$。$\cdots$,将 $[a_{k-1}, b_{k-1}]$ 均分为
\[
\left[ a_{k-1}, \frac{a_{k-1} + b_{k-1}}{2} \right], \left[ \frac{a_{k-1} + b_{k-1}}{2}, b_{k-1} \right],
\]
则必有其中之一闭区间含有 $S$ 中无穷个点,记此闭区间为 $[a_k, b_k]$。如此一直下去,得到一列闭区间 $\{[a_n, b_n]\}$,且
\[
\lim_{n \to \infty} (b_n - a_n) = \lim_{n \to \infty} \frac{1}{2^n} (b - a) = 0,
\]
因此 $\{[a_n, b_n]\}$ 形成一个闭区间套。根据 \textcolor{red}{闭区间套定理},存在唯一的实数 $\xi \in \mathbb{R}$,使得
\[
a_n \leq \xi \leq b_n,\quad n = 1, 2, 3, \ldots,
\]
且
\[
\lim_{n \to \infty} a_n = \lim_{n \to \infty} b_n = \xi。
\]

下说明 $\xi$ 是 $S$ 的一个聚点。在每个 $[a_n, b_n]$ 中取一个 $S$ 中的元素记为 $x_n$,易知 $\lim_{n \to \infty} x_n = \alpha$,因此 $\alpha$ 是 $S$ 的聚点,证毕。


\section*{闭区间套定理 $\Rightarrow$ 有限覆盖定理}

\textbf{证明:} 设闭区间 $[a, b]$ 在实数 $R$ 上,任取闭区间 $[a, b]$ 的一个开覆盖 $\{O_{\lambda}\}$,反证法,假设闭区间 $[a, b]$ 不能被 $\{O_{\lambda}\}$ 的有限个子集覆盖,将 $[a, b]$ 二等分为
\[
\left[ a, \frac{a + b}{2} \right], \left[ \frac{a + b}{2}, b \right],
\]
至少有其中之一不能被 $\{O_{\lambda}\}$ 的有限个子集覆盖,将此区间记为 $[a_1, b_1]$。将 $[a_1, b_1]$ 二等分为
\[
\left[ a_1, \frac{a_1 + b_1}{2} \right], \left[ \frac{a_1 + b_1}{2}, b_1 \right],
\]
至少有其中之一不能被 $\{O_{\lambda}\}$ 的有限个子集覆盖,将此区间记为 $[a_2, b_2]$。如此下去,便可得到一列闭区间 $\{[a_n, b_n]\}$,且
\[
\lim_{n \to \infty} (b_n - a_n) = \lim_{n \to \infty} \left(\frac{1}{2}\right)^n (b - a) = 0。
\]

因此 $\{[a_n, b_n]\}$ 形成一个闭区间套。根据 \textcolor{red}{闭区间套定理},存在唯一的实数 $\xi$ 属于所有的闭区间 $[a_n, b_n]$,且
\[
\xi = \lim_{n \to \infty} a_n = \lim_{n \to \infty} b_n, \quad a_n \leq \xi \leq b_n,\quad n = 1, 2, 3, \ldots。
\]

于是存在 $\{O_{\lambda}\}$ 的一个开区间 $O_{\lambda^*}$,使得 $\xi \in O_{\lambda^*}$,当 $n$ 充分大时,必有 $[a_n, b_n] \subseteq O_{\lambda^*}$ 与 $[a, b]$ 不能被 $\{O_{\lambda}\}$ 的有限个子集覆盖矛盾。因此存在 $\{O_{\lambda}\}$ 的有限个子集覆盖闭区间 $[a, b]$,证毕。


\begin{theorem}[ 聚点定理] \label{tabe222}
(1)在一个有界的无限点集 $S$ 中,至少存在一个实数 $\xi$,它是这个集合的聚点。即在这个点的任意小的邻域内,总是包含这个集合中的无穷多个点。

   (2) 设 $S$ 是实数集合 $\mathbb{R}$ 的一个有界无限点集,则存在一个实数 $\xi \in \mathbb{R}$,使得对于任意的 $\varepsilon > 0$,区间 $(\xi - \varepsilon, \xi + \varepsilon)$ 内总是包含 $S$ 中的无穷多个点。

(3)\  \begin{itemize}
      \item 聚点定理:在一个有界序列中,至少存在一个聚点。
      \item 魏尔斯特拉斯定理:每个有界数列都有至少一个聚点。
      \item 致密性定理:一个集合是致密的当且仅当它的每一个开覆盖都有有限子覆盖。
      \item Bolzano-Weierstrass定理:任何有界数列必有一个收敛的子序列。
    \end{itemize}

\end{theorem}



\section*{聚点定理 $\Rightarrow$ 确界存在定理}

\textbf{证明:} 设 $S$ 是非空有上界数集,任取它的一个上界 $b \notin S$,取 $a \in S$,使得 $S \cap (a, b) \neq \emptyset$。对 $[a, b]$ 一等分为
\[
\left[a, \frac{a + b}{2}\right] \quad \text{和} \quad \left[\frac{a + b}{2}, b\right],
\]
若 $\left[a, \frac{a + b}{2}\right] \cap S \neq \emptyset$,则记 $[a_1, b_1] = \left[a, \frac{a + b}{2}\right]$,否则记 $[a_1, b_1] = \left[\frac{a + b}{2}, b\right]$。对 $[a_1, b_1]$ 一等分为
\[
\left[a_1, \frac{a_1 + b_1}{2}\right] \quad \text{和} \quad \left[\frac{a_1 + b_1}{2}, b_1\right],
\]
若 $\left[a_1, \frac{a_1 + b_1}{2}\right] \cap S \neq \emptyset$,则记 $[a_2, b_2] = \left[a_1, \frac{a_1 + b_1}{2}\right]$,否则记 $[a_2, b_2] = \left[\frac{a_1 + b_1}{2}, b_1\right]$。依此类推,对 $[a_{n-1}, b_{n-1}]$ 一等分为
\[
\left[a_{n-1}, \frac{a_{n-1} + b_{n-1}}{2}\right] \quad \text{和} \quad \left[\frac{a_{n-1} + b_{n-1}}{2}, b_{n-1}\right],
\]
若 $\left[a_{n-1}, \frac{a_{n-1} + b_{n-1}}{2}\right] \cap S \neq \emptyset$,则记 $[a_n, b_n] = \left[a_{n-1}, \frac{a_{n-1} + b_{n-1}}{2}\right]$,否则记 $[a_n, b_n] = \left[\frac{a_{n-1} + b_{n-1}}{2}, b_{n-1}\right]$。依此类推,得到一列闭区间 $\{[a_n, b_n]\}$。记 $R = \{b_1, b_2, \ldots, b_n, \ldots\}$,其中如有相同项则记一个。易知集合 $R$ 有界,根据聚点定理,集合 $R$ 必有聚点记为 $\xi$,则有 $\{b_{n_k}\}$ 的子列 $\lim_{k \to \infty} b_{n_k} = \xi$。

下说明 $\xi$ 是 $S$ 的上确界:
\begin{enumerate}
    \item $\forall x \in S$,有 $x \leq \xi$;
    \item $\forall \varepsilon > 0$,$\exists N \in \mathbb{N}^+$,使得 $\xi - a_n < \varepsilon$,即 $b < \xi + \varepsilon$,其中 $a_n \in S$。
\end{enumerate}
\begin{change}
\item 由 $|a_n - b_{n_k}| = \frac{1}{2^{n_k}} (b - a)$ 知,$\lim_{n \to \infty} a_n = \xi$;
\item $\xi$ 是 $S$ 的一个上界,不然存在 $x_0 \in S$,使得 $x_0 > \xi$,由 $\lim_{k \to \infty} b_{n_k} = \xi$ 知,对于 $\varepsilon = x_0 - \xi$,$\exists K \in \mathbb{N}^+$,$\forall k > K$ 有 $-\varepsilon < b_{n_k} - \xi < \varepsilon$,即 $b_{n_k} - \xi < \varepsilon$,因此 $x_0 > b_{n_k}$ 是 $S$ 的上界矛盾,从而 $\xi$ 是 $S$ 的一个上界;
 \item $\xi$ 是 $S$ 的最小上界,即 $\forall \varepsilon > 0$,$\exists K \in \mathbb{N}^+$,使得 $b_{n_k} - \varepsilon < \xi < \varepsilon$,因为 $\lim_{k \to \infty} b_{n_k} = \xi$,所以 $\forall \varepsilon > 0$,$\exists K \in \mathbb{N}^+$,使得 $-\varepsilon < b_{n_k} - \xi < \varepsilon$,从而有 $b_{n_k} - \varepsilon < \xi < \varepsilon$。

综合①②③得 $\xi$ 是 $S$ 的上确界。
\end{change}


\section*{聚点定理 $\Rightarrow$ 单调有界定理}

\textbf{证明:} 设数列 $\{x_n\}$ 是单调增加有上界,令 $S = \{x_1, x_2, \ldots\}$,任取它的一个上界 $b \notin S$,任取 $a \in S$,则 $S \subseteq [a, b]$。对 $[a, b]$ 二等分为
\[
\left[ a, \frac{a + b}{2} \right], \left[ \frac{a + b}{2}, b \right],
\]
若 $\frac{a + b}{2}$ 非 $S$ 的上界,则记 $[a_1, b_1] = \left[ a, \frac{a + b}{2} \right]$,否则记 $[a_1, b_1] = \left[ \frac{a + b}{2}, b \right]$。对 $[a_1, b_1]$ 二等分为
\[
\left[ a_1, \frac{a_1 + b_1}{2} \right], \left[ \frac{a_1 + b_1}{2}, b_1 \right],
\]
若 $\frac{a_1 + b_1}{2}$ 非 $S$ 的上界,则记 $[a_2, b_2] = \left[ a_1, \frac{a_1 + b_1}{2} \right]$,否则记 $[a_2, b_2] = \left[ \frac{a_1 + b_1}{2}, b_1 \right]$。依此类推,对 $[a_{n-1}, b_{n-1}]$ 二等分为
\[
\left[ a_{n-1}, \frac{a_{n-1} + b_{n-1}}{2} \right], \left[ \frac{a_{n-1} + b_{n-1}}{2}, b_{n-1} \right],
\]
若 $\frac{a_{n-1} + b_{n-1}}{2}$ 非 $S$ 的上界,则记 $[a_n, b_n] = \left[ a_{n-1}, \frac{a_{n-1} + b_{n-1}}{2} \right]$,否则记 $[a_n, b_n] = \left[ \frac{a_{n-1} + b_{n-1}}{2}, b_{n-1} \right]$。依此类推,得到一列闭区间 $\{[a_n, b_n]\}$。易知 $\{[a_n, b_n]\}$ 满足:
\begin{enumerate}
    \item $\{[a_{n+1}, b_{n+1}] \subseteq [a_n, b_n]\}$ 对于所有正整数 $n$ 都成立;
    \item $\lim_{n \to \infty} (b_n - a_n) = \lim_{n \to \infty} \left(\frac{1}{2}\right)^n (b - a) = 0$。
\end{enumerate}

因此,$\{[a_n, b_n]\}$ 形成一个闭区间套。根据 \textcolor{red}{聚点定理},$\{[a_n, b_n]\}$ 每一个区间都至少有一个数列 $\{x_n\}$ 的聚点。又数列 $\{x_n\}$ 单调增加,且 $b_n - a_n = \frac{1}{2^n} (b - a)$。因此数列 $\{x_n\}$ 聚点唯一,设为 $\xi$,于是有 $\lim_{n \to \infty} x_n = \xi$。证毕。

\section*{聚点定理 $\Rightarrow$ Cauchy 收敛准则}

\textbf{证明:} 先证数列 $\{x_n\}$ 收敛 $\Rightarrow$ 数列 $\{x_n\}$ 是基本数列。设 $\lim_{n \to \infty} x_n = a$,则 $\forall \varepsilon > 0$,$\exists N \in \mathbb{N}^+$,使得
\[
\forall n, m > N,\, 都有 \, |x_n - a| < \frac{\varepsilon}{2},\, |x_m - a| < \frac{\varepsilon}{2},
\]
从而
\[
|x_n - x_m| \leq |x_n - a| + |x_m - a| < \varepsilon,
\]
因此数列 $\{x_n\}$ 是基本数列。

再证数列 $\{x_n\}$ 是基本数列 $\Rightarrow$ 数列 $\{x_n\}$ 收敛。设数列 $\{x_n\}$ 是基本数列,下说明数列 $\{x_n\}$ 有界。$\forall \varepsilon > 0$,$\exists N \in \mathbb{N}^+$,$\forall n, m > N$,都有 $|x_n - x_m| < \varepsilon$。取定 $\varepsilon = 1$ 时,存在 $N_1 \in \mathbb{N}^+$,使得
\[
\forall n > N_1, \, |x_n - x_{N_1}| < \varepsilon,
\]
于是 $M = \max \{x_1, x_2, \ldots, x_{N_1}, |x_{N_1}| + 1\}$ 是数列 $\{x_n\}$ 的一个上界,
\[
m = \min \{x_1, x_2, \ldots, x_{N_1}, -1\}
\]
是数列 $\{x_n\}$ 的一个下界,因此数列 $\{x_n\}$ 有界。下说明数列 $\{x_n\}$ 收敛。将 $[m, M]$ 均分为
\[
\left[m, \frac{m + M}{2}\right], \left[\frac{m + M}{2}, M\right],
\]
则必有其中之一闭区间含有 $\{x_n\}$ 中无穷个点,记此闭区间为 $[m_1, M_1]$。将 $[m_1, M_1]$ 均分为
\[
\left[m_1, \frac{m_1 + M_1}{2}\right], \left[\frac{m_1 + M_1}{2}, M_1\right],
\]
则必有其中之一闭区间含有 $\{x_n\}$ 中无穷个点,记此闭区间为 $[m_2, M_2]$。$\cdots$,将 $[m_{k-1}, M_{k-1}]$ 均分为
\[
\left[m_{k-1}, \frac{m_{k-1} + M_{k-1}}{2}\right], \left[\frac{m_{k-1} + M_{k-1}}{2}, M_{k-1}\right],
\]
则必有其中之一闭区间含有 $\{x_n\}$ 中无穷个点,记此闭区间为 $[m_k, M_k]$。如此一直下去,得到一列闭区间 $\{[m_n, M_n]\}$,且
\[
\lim_{n \to \infty} (M_n - m_n) = \lim_{n \to \infty} \frac{1}{2^n} (M - m) = 0,
\]
易知 $\{m_n\}$ 单调增加,数列 $\{M_n\}$ 单调减少,根据 \textcolor{red}{聚点定理},存在实数 $\alpha, \beta$ 使得
\[
\lim_{n \to \infty} m_n = \alpha, \quad \lim_{n \to \infty} M_n = \beta,
\]
且
\[
m_n \leq \alpha \leq \beta \leq M_n, \quad n = 1, 2, 3, \ldots
\]
可知 $\lim_{n \to \infty} m_n = \alpha = \beta = \lim_{n \to \infty} M_n$。再根据数列 $\{x_n\}$ 是基本数列 $\Rightarrow$ 数列 $\{x_n\}$ 收敛,设数列 $\{x_n\}$ 是基本数列,下说明数列 $\{x_n\}$ 有界。由此得证。


\section*{聚点定理 $\Rightarrow$ 闭区间套定理}

\textbf{证明:} 设一列闭区间 $\{[a_n, b_n]\}$ 形成一个闭区间套,则数列 $\{a_n\}$, $\{b_n\}$ 有界,存在性,根据 \textcolor{red}{聚点定理},存在实数 $\alpha, \beta$,子列 $\{a_{n_k}\} \subseteq \{a_n\}$,子列 $\{b_{n_k}\} \subseteq \{b_n\}$,使得 $\lim_{k \to \infty} a_{n_k} = \alpha$,$\lim_{k \to \infty} b_{n_k} = \beta$。

由数列 $\{a_n\}$ 单调递增有上界,数列 $\{b_n\}$ 单调递减有下界知 $\lim_{n \to \infty} a_n = \alpha$,$\lim_{n \to \infty} b_n = \beta$,且 $\alpha \leq a_n \leq b_n \leq \beta$,$n = 1, 2, 3, \ldots$。又 $\lim_{n \to \infty} (b_n - a_n) = 0$,于是
\[
\lim_{n \to \infty} a_n = \alpha = \beta = \lim_{n \to \infty} b_n。
\]

唯一性:若另有 $\alpha'$,使得 $a_n \leq \alpha' \leq b_n,n = 1, 2, 3, \ldots$ 且 $\lim_{n \to \infty} a_n = \alpha',$由 $\lim_{n \to \infty} (b_n - a_n) = 0$,可知 $\alpha = \alpha'$。证毕。


\section*{聚点定理 $\Rightarrow$ 有限覆盖定理}

\textbf{证明:} 设闭区间 $[a, b]$ 在实数 $R$ 上,任取闭区间 $[a, b]$ 的一个开覆盖 $\{O_{\lambda}\}$,反证法,假设闭区间 $[a, b]$ 不能被 $\{O_{\lambda}\}$ 的有限个子集覆盖,将 $[a, b]$ 二等分为
\[
\left[ a, \frac{a + b}{2} \right], \left[ \frac{a + b}{2}, b \right],
\]
至少有其中之一不能被 $\{O_{\lambda}\}$ 的有限个子集覆盖,将此区间记为 $[a_1, b_1]$。将 $[a_1, b_1]$ 二等分为
\[
\left[ a_1, \frac{a_1 + b_1}{2} \right], \left[ \frac{a_1 + b_1}{2}, b_1 \right],
\]
至少有其中之一不能被 $\{O_{\lambda}\}$ 的有限个子集覆盖,将此区间记为 $[a_2, b_2]$。如此下去,便可得到一列闭区间 $\{[a_n, b_n]\}$,且
\[
\lim_{n \to \infty} (b_n - a_n) = \lim_{n \to \infty} \left(\frac{1}{2}\right)^n (b - a) = 0。
\]

易知数列 $\{a_n\}$ 单调增加有上界,数列 $\{b_n\}$ 单调减少有下界。根据 \textcolor{red}{聚点定理},存在实数 $\alpha$,使得
\[
\lim_{n \to \infty} a_n = \alpha, \quad \lim_{n \to \infty} b_n = \alpha。
\]

于是存在 $\{O_{\lambda}\}$ 的一个开区间 $O_{\lambda^*}$,使得 $\alpha \in O_{\lambda^*}$,当 $n$ 充分大时,必有 $[a_n, b_n] \subseteq O_{\lambda^*}$,与 $[a_n, b_n]$ 不能被 $\{O_{\lambda}\}$ 的有限个子集覆盖矛盾。因此存在 $\{O_{\lambda}\}$ 的有限个子集覆盖闭区间 $[a, b]$,证毕。




\begin{theorem}[ 有限覆盖定理] \label{ pro:js5}
(1)一个集合是紧的当且仅当它是闭且有界的。

(2)若一个开覆盖 \( S \) 覆盖了闭区间 \([a, b]\),即 \([a, b] \subseteq \bigcup_{i \in I} U_i\),其中 \( U_i \) 是 \( S \) 中的开区间,则存在有限个开区间 \( U_{i_1}, U_{i_2}, \ldots, U_{i_n} \) 使得 \([a, b] \subseteq \bigcup_{j=1}^n U_{i_j}\)。

\end{theorem}


\section*{有限覆盖定理 $\Rightarrow$ 确界存在定理}

\textbf{证明:} 反证法,设 $S$ 是非空有上界的实数集,假设 $S$ 没有上确界(最小上界)。任取它的一个上界 $b \notin S$,任取 $a \in S$,则 $\forall x \in [a, b]$ 有:
\begin{enumerate}
    \item $x$ 是 $S$ 的一个上界,由于 $S$ 无最小上界,因此 $\exists x' \in [a, b]$,$x'$ 是 $S$ 的一个上界,且 $x' < x$,从而存在 $x$ 的一个开邻域 $O_x$,其中 $O_x$ 中的元素都是 $S$ 的上界;
    \item $x$ 不是 $S$ 的上界,则存在 $x'' \in S \cap [a, b],且 x'' > x$,于是存在 $x$ 的一个开邻域 $O_x$,其中 $O_x$ 中的元素都不是 $S$ 的上界。
\end{enumerate}

因此 $[a, b]$ 上的每个点都能找到一个开邻域 $O_x$,它要么属于第一类(每个点都为 $S$ 的上界),要么属于第二类(每个点都不是 $S$ 的上界),从而 $\{O_x \mid x \in [a, b]\}$ 是 $[a, b]$ 的一个开覆盖。根据 \textcolor{red}{有限覆盖定理},存在有限子集 $\{O_{x_1}, O_{x_2}, \ldots, O_{x_k}\}$,使得 $[a, b] \subseteq \bigcup_{i=1}^{k} O_{x_i}$。又因为 $b$ 所在的开区间属于第一类,相邻接的开区间有公共点,也应为第一类的,经过有限邻接,可知 $a$ 所在的开区间也是第一类的,与 $a$ 不是 $S$ 的上界矛盾,因此 $S$ 有上确界。证毕。



\section*{有限覆盖定理 $\Rightarrow$ 单调有界定理}

\textbf{证明:} 用反证法,设数列 $\{x_n\}$ 单调递增有上界,于是存在 $[a, b] \subseteq \mathbb{R}$ 使得 $\forall n \in \mathbb{N}^+$,$x_n \in [a, b]$。假设 $\{x_n\}$ 不收敛,则 $\forall x \in [a, b]$,必存在 $x$ 的某个开邻域 $O_x$ 只含有数列 $\{x_n\}$ 的有限个点。令 $S = \{O_x \mid x \in [a, b]\}$,则 $S$ 是 $[a, b]$ 的一个开覆盖。根据 \textcolor{red}{有限覆盖定理},$S$ 的有限子集 $\{O_1, O_2, \ldots, O_k\}$ 可以覆盖 $[a, b]$,即 $[a, b] \subseteq \bigcup_{i=1}^{k} O_i$。但每个 $O_i$ 中只能含有数列 $\{x_n\}$ 的有限个点,与 $\{x_n\}$ 在 $[a, b]$ 中齐矛盾,因此数列 $\{x_n\}$ 必收敛。证毕。

\section*{有限覆盖定理 $\Rightarrow$ Cauchy 收敛准则}

\textbf{证明:} 先证数列 $\{x_n\}$ 收敛 $\Rightarrow$ 数列 $\{x_n\}$ 是基本数列。设 $\lim_{n \to \infty} x_n = a$,则 $\forall \varepsilon > 0$,$\exists N \in \mathbb{N}^+$,使得
\[
\forall n, m > N, \text{都有} |x_n - a| < \frac{\varepsilon}{2}, |x_m - a| < \frac{\varepsilon}{2},
\]
从而
\[
|x_n - x_m| \leq |x_n - a| + |x_m - a| < \varepsilon,
\]
因此数列 $\{x_n\}$ 是基本数列。

再证数列 $\{x_n\}$ 是基本数列 $\Rightarrow$ 数列 $\{x_n\}$ 收敛。设数列 $\{x_n\}$ 是基本数列,下说明数列 $\{x_n\}$ 有界。$\forall \varepsilon > 0$,$\exists N \in \mathbb{N}^+$,$\forall n, m > N$,都有 $|x_n - x_m| < \varepsilon$。取定 $\varepsilon = 1$ 时,存在 $N_1 \in \mathbb{N}^+$,使得
\[
\forall n > N_1, |x_n - x_{N_1}| < \varepsilon,
\]


于是 $M = \max \{x_1, x_2, \ldots, x_{N_1}, |x_{N_1}| + 1\}$ 是数列 $\{x_n\}$ 的一个上界,
\[
m = \min \{x_1, x_2, \ldots, x_{N_1}, -1\}
\]
是数列 $\{x_n\}$ 的一个下界,因此数列 $\{x_n\}$ 有界。下说明数列 $\{x_n\}$ 收敛。假设数列 $\{x_n\}$ 不存在收敛子列,则 $\forall x \in [m, M]$,存在 $x$ 的一个开邻域 $O_x$,使得 $O_x$ 至多含有数列 $\{x_n\}$ 的有限项,令 $D = \{O_x \mid x \in [m, M]\}$,则 $D$ 是 $[m, M]$ 的一个开覆盖。根据 \textcolor{red}{有限覆盖定理},存在 $D$ 的有限个子集 $\{O_{x_1}, O_{x_2}, \ldots, O_{x_k}\}$,使得 $[m, M] \subseteq \bigcup_{i=1}^{k} O_{x_i}$。而 $\bigcup_{i=1}^{k} O_{x_i}$ 中只含有数列 $\{x_n\}$ 的有限项,与 $[m, M] \subseteq \bigcup_{i=1}^{k} O_{x_i}$ 矛盾,因此数列 $\{x_n\}$ 存在收敛子列。设存在实数 $\xi$,数列 $\{x_n\}$ 的一个子列 $\{x_{n_k}\}$,使得
\[
\lim_{k \to \infty} x_{n_k} = \xi,
\]
又数列 $\{x_n\}$ 是基本数列,则
\[
\forall \varepsilon > 0, \exists N \in \mathbb{N}^+, \forall n, m > N, |x_n - x_m| < \frac{\varepsilon}{2},
\]
而根据 $\lim_{k \to \infty} x_{n_k} = \xi$ 有
\[
\forall \varepsilon > 0, \exists N \in \mathbb{N}^+, n_k > N, k > K,有 |x_{n_k} - \xi| < \frac{\varepsilon}{2}。
\]
于是
\[
\forall \varepsilon > 0, \forall n > N, k > K 有 |x_n - \xi| \leq |x_n - x_{n_k}| + |x_{n_k} - \xi| < \varepsilon。
\]
因此数列 $\{x_n\}$ 收敛。证毕。



\section*{有限覆盖定理 $\Rightarrow$ 闭区间套定理}

\textbf{证明:} 设一列闭区间 $\{[a_n, b_n]\}$ 形成一个闭区间套,存在性,下说明 $\bigcap_{n=1}^{\infty} [a_n, b_n] \neq \varnothing$。假设 $\bigcap_{n=1}^{\infty} [a_n, b_n] = \varnothing$,则 $\forall x \in [a, b]$,存在一个开邻域 $O_x$,使得 $O_x$ 不全与 $[a_n, b_n]$ 相交,即存在 $N \in \mathbb{N}^+$,使得 $O_x \cap [a_N, b_N] = \varnothing$。令 $S = \{O_x \mid x \in [a, b]\}$,于是 $S$ 是 $[a, b]$ 的一个开覆盖。根据 \textcolor{red}{有限覆盖定理},存在 $S$ 的有限个子集 $\{O_{x_1}, O_{x_2}, \ldots, O_{x_k}\}$,使得 $[a, b] \subseteq \bigcup_{i=1}^{k} O_{x_i}$。而 $\bigcap_{n=1}^{\infty} [a_n, b_n] = \varnothing$,$i=1,2,\ldots,k$。取 $[a_{n_i}, b_{n_i}]$ 中最小的区间为 $[a_0, b_0]$,于是
\[
[a_0, b_0] \cap \bigcup_{i=1}^{k} O_{x_i} = \varnothing
\]
与 $[a_0, b_0] \subseteq \bigcup_{i=1}^{k} O_{x_i}$ 矛盾。因此 $\bigcap_{n=1}^{\infty} [a_n, b_n] \neq \varnothing$。即存在 $\xi \in \bigcap_{n=1}^{\infty} [a_n, b_n]$。由于 $\{[a_n, b_n]\}$ 为闭区间套,得 $\lim_{n \to \infty} (b_n - a_n) \to 0$,故 $\xi$ 唯一。若另有 $\xi' \in \bigcap_{n=1}^{\infty} [a_n, b_n]$,则由 $|\xi - \xi'| \leq |b_n - a_n| \to 0$($n \to \infty$)知,$\xi = \xi'$。于是 $\xi = \bigcap_{n=1}^{\infty} [a_n, b_n]$。证毕。


\section*{有限覆盖定理 $\Rightarrow$ 聚点定理}

\textbf{证明:} 设 $S$ 是有界无穷点集,则存在闭区间 $[a, b]$,使得 $S \subseteq [a, b]$。

情形一,存在 $\xi \in [a, b]$,$\xi$ 的任意邻域都包含 $S$ 中的无限项,则在开区间 $\left(\xi - \frac{1}{k}, \xi + \frac{1}{k}\right)$ 中任取 $S$ 中的一项记为 $x_k$,其中 $x_k$ 取与 $x_{k-1}$ 相异的项,$k = 1, 2, 3, \ldots$,于是 $\lim_{n \to \infty} x_n = \xi$,因此 $\xi$ 为数集 $S$ 的聚点;

情形二,不存在情形一中的 $\xi$,即 $\forall x \in [a, b]$,存在 $x$ 的一个开邻域 $O_x$,使得 $O_x$ 只含有数集 $S$ 中的有限项。令 $D = \{O_x \mid x \in [a, b]\}$,则 $D$ 是 $[a, b]$ 的一个开覆盖。根据 \textcolor{red}{有限覆盖定理},存在 $D$ 的有限个子集 $\{O_{x_1}, O_{x_2}, \ldots, O_{x_k}\}$,使得 $[a, b] \subseteq \bigcup_{i=1}^{k} O_{x_i}$,而 $\bigcup_{i=1}^{k} O_{x_i}$ 中只含有数集 $S$ 中的有限项,与 $S \subseteq [a, b] \subseteq \bigcup_{i=1}^{k} O_{x_i}$ 矛盾,于是只可能情形一成立,因此数集 $S$ 必有聚点,证毕。

\chapter{数学分析、实分析、泛函分析}
\section*{Heine-Borel 定理}
在实数空间 $\mathbb{R}^n$ 中,一个子集 $S \subseteq \mathbb{R}^n$ 是紧的,当且仅当它是闭且有界的。

\t{证明}

\subsection*{必要性}
如果 $S$ 是紧的,那么 $S$ 是闭且有界的。

\subsubsection*{有界性}
假设 $S$ 是紧的。若 $S$ 不有界,那么对于每个正整数 $k$,存在 $x_k \in S$,使得 $\|x_k\| \ge k$。这意味着我们可以在 $S$ 中找到一个无穷序列 $\{x_k\}$ 使得 $\|x_k\| \to \infty$。但由于 $S$ 是紧的,$\{x_k\}$ 必须有一个收敛的子列。然而,这个子列的极限必须在 $S$ 中,并且它的极限必定是无穷大的,这与 $S$ 是紧的矛盾。所以 $S$ 必须是有界的。

\subsubsection*{闭性}
假设 $S$ 是紧的。如果 $S$ 不是闭的,那么存在一个点 $x \in \overline{S} \setminus S$,其中 $\overline{S}$ 表示 $S$ 的闭包。由于 $x \notin S$,但 $x \in \overline{S}$,所以存在 $S$ 中的一个序列 $\{x_k\}$ 收敛于 $x$。由于 $S$ 是紧的,$\{x_k\}$ 的收敛极限 $x$ 必须在 $S$ 中,这与 $x \notin S$ 矛盾。因此,$S$ 必须是闭的。

\subsection*{充分性}
如果 $S$ 是闭且有界的,那么 $S$ 是紧的。

根据 Bolzano-Weierstrass 定理,任何在 $\mathbb{R}^n$ 中的有界序列都有一个收敛子列。因为 $S$ 是有界的,所以 $S$ 中的任意序列必然有一个有界子序列。根据 Bolzano-Weierstrass 定理,这个有界子序列有一个收敛子列。

我们需要证明这个收敛子列的极限点也在 $S$ 中。因为 $S$ 是闭的,所以所有极限点都在 $S$ 中。于是,这个收敛子列的极限点在 $S$ 中,这意味着 $S$ 是序列紧的。

在 $\mathbb{R}^n$ 中,序列紧性和覆盖紧性等价。因此,$S$ 是覆盖紧的。

\section*{结论}
综上所述,我们证明了在 $\mathbb{R}^n$ 中,Heine-Borel 定理成立:一个子集 $S \subseteq \mathbb{R}^n$ 是紧的,当且仅当它是闭且有界的。


\newpage

任意数列都存在单调子列。

\section*{证明}
设数列为 $\{a_n\}$,下面分两种情形来讨论:

1. 若对任何正整数 $k$,数列 $\{a_{k+n}\}$ 有最大项。设 $\{a_{k+n}\}$ 的最大项为 $a_{n_1}$,因 $\{a_{n_1+n}\}$ 亦有最大项,设其最大项为 $a_{n_2}$,显然有 $n_2 > n_1$,且因 $a_{n_1}$ 是 $\{a_{n_1+n}\}$ 的一个子列,故
   \[
   a_{n_2} \leq a_{n_1};
   \]
   同理存在 $n_3 > n_2$,使得
   \[
   a_{n_3} \leq a_{n_2};
   \]
   \[
   \vdots
   \]
   这样就得到一个单调递减的子列 $\{a_{n_k}\}$。

2. 至少存在某正整数 $k$,数列 $\{a_{k+n}\}$ 没有最大项。先取 $n_1 = k+1$,因 $\{a_{k+n}\}$ 没有最大项,故 $a_{n_1}$ 后面总存在项 $a_{n_2} (n_2 > n_1)$,使得
   \[
   a_{n_2} > a_{n_1};
   \]
   同理在 $a_{n_2}$ 后面的项 $a_{n_3} (n_3 > n_2)$,使得
   \[
   a_{n_3} > a_{n_2};
   \]
   \[
   \vdots
   \]
   这样就得到一个严格递增的子列 $\{a_{n_k}\}$。

\chapter{代数、几何、微分方程、拓扑、分数阶微分方程}

敬请期待$\ldots$


\end{document}